\documentclass{article}

\usepackage{listings}
\usepackage{multicol}
\usepackage{graphicx}
\usepackage[
	colorlinks = true,
	linkcolor = blue,
	urlcolor = blue,
	citecolor = blue,
	anchorcolor = blue
]{hyperref}
\usepackage{parskip}


\title{Finding Architectural Knowledge in Emails and Issues}
\author{Andrew Lalis}

\begin{document}

\maketitle
\section{Abstract}
	This is the abstract!

\newpage

\tableofcontents

\section{Introduction}
	This is a thesis about finding architectural knowledge in emails! Wow!

\section{Categorization of Architectural Knowledge}
	According to Tom's research \cite{denBoon21} we know lots about architecture.

\section{Preparing Data for Analysis}
	While den Boon's ArchDetector web application provided a solid foundation for extracting normalized data from various Apache mailing lists (in the form of \textit{mbox} files) and Jira issues\cite{denBoon21}, it was not designed with the foresight to accommodate further developments, so a new, simple command-line application was developed to improve collaborative workflow of categorizing and exporting data.
	
	This application is called \textbf{ak-tagger}, and its source and usage documentation may be viewed online at \href{https://github.com/andrewlalis/ArchitecturalKnowledgeResearch}{this paper's associated GitHub repository}. A short description of its functionality will be supplied inline with this paper, but please refer to the online documentation for any further information or to report issues.
	
	\subsection{About the ak-tagger Program}
		The architectural knowledge tagger program, colloquially known as \textbf{ak-tagger}, functions as a companion to the ArchDetector web application, that provides a more efficient interface for tagging and evaluating the categorization of sources of architectural knowledge, and sharing this information with others. The program is written in the \href{https://dlang.org/}{D programming language}, and can be compiled to a native executable with any D compiler (DMD, GDC, LDC) that provides the \textbf{dub} build tool.
		
		The ak-tagger program works with raw JSON data sets that are grouped into files based on the query that produced the set of results. Thus the program is designed primarily as a tool for analysis of prepared search queries and their results, as obtained from an instance of the ArchDetector REST API.
		
		\begin{figure}
			\includegraphics[width=\textwidth]{./images/ak-tagger_system_diagram.png}
			\caption{System diagram for the ak-tagger program's data usage.}
		\end{figure}
		
		There program has several distinct functionalities:
		\begin{enumerate}
			\item \textbf{Fetch} - The user is able to fetch a raw dataset from the ArchDetector API and save it to a JSON file for sharing or their own use.
			\item \textbf{Use} - Fetched datasets may be explored and categorized with a simple interactive command-line interface. Offers utilities such as exporting email threads as formatted text files for easy viewing, and modifying an email thread's set of tags.
			\item \textbf{Inspect} - Datasets may be inspected to produce formatted aggregate information that's easier to use than the entire raw JSON file, including some aggregate information.
		\end{enumerate}
	
		The general workflow for a user is to first \textit{fetch} a dataset from ArchDetector, then save it as a JSON file, then \textit{use} it to categorize email threads according to their architectural knowledge content, and finally it will be used to
		

% The bibliography for this thesis.
\bibliographystyle{plain}
\begin{thebibliography}{99}
	\bibitem{denBoon21}
		Tom den Boon.
		Exploring the effectiveness of search engines for finding architectural knowledge in open source repositories.
		\textit{University of Groningen, Faculty of Science and Engineering Student Theses}.
		2021.
	\bibitem{kruchten06}
		Kruchten P., Lago P., van Vliet H. (2006) Building Up and Reasoning About Architectural Knowledge. In: Hofmeister C., Crnkovic I., Reussner R. (eds) Quality of Software Architectures. QoSA 2006. Lecture Notes in Computer Science, vol 4214. Springer, Berlin, Heidelberg. https://doi.org/10.1007/11921998\_8
\end{thebibliography}

\end{document}
